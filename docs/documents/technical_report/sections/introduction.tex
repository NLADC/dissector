\section{Motivation}


Internet and our increasing dependency...
DDoS attacks is a serious problem NUMBERS...
DDoS attacks are old and still growing...
DDoS attacks will not disappear. Compare to other problems: computer virus, spam, phishing, ransom-ware will also not disappear BUT still we can control the problem!

Enable people to go further!

Going distributed!
  
DDoSDB is a very ambitious idea for helping to control the DDoS attack problem worldwide. For achieving such ambitious goal, DDoSDB is focussed on facilitating sharing real attack data and producing generic fingerprints to improve the detection and mitigation of attacks.

benefit everyone! 

Why is DDoSDB unique? 
\begin{itemize}
	\item No conflict of interest;
	\item Non-profit; 
	\item 
\end{itemize}



guaranteeing the privacy of 

It is a public, open, collaborative, and free of charge solution. The most ikey element of DDoSDB is . Based on the collected data, DDoSDB makes available generic and specific fingerprints for a wide-range of purposes:

DDoSDB does NOT belong a specify group of people but to anyone willingly to help controlling the DDoS attack problem, and protect themselves. 

used to improve detection and mitigation solutions against DDoS attacks.

Enforce other great solutions: Ampot, Hopscotch, 

I am not against the large security companies that protect organizations against attacks. We believe that the key element for resolving the problem is 'Real attack data'. The main problem 
attack data is how the l 

Attack vector is a set of characteristics that define an attack. 

Why post-mortem? it gives the guarantee that there is at least one attack vector within a data collected by a target/victim of an attack 


Post-mortem Analysis for Improving DDoS Attack Detection and Mitigation


From only post-mortem to a fully automated solution!

\begin{itemize}
	\item Phase 1: Create modules for facilitating DDoS attack data sharing (by victims);
	\item Phase 2: Develop modules for automatic generation of detection and mitigation rules and signatures for the most used technologies;
	\item Phase 3: Phase 3: Anomaly detection, notification, and distributed detection and mitigation;
	\item Phase 4:
	\item Phase 5:
\end{itemize}

Phase 1 is the most important because it enables the creation of a database with real attacks! This attacks will serve for:
\begin{itemize}
	\item comparison of 
	\item creation of generic and specific fingerprints;
	\item attribution and prosecution of attackers;
	\item data correlation with other very important databases;
	\item teach on how to replay and attacks
\end{itemize}


How to participate?
provide DDoS attack data!


Disclaimer: this is a non-peer reviewed!